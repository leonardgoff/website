\documentclass{article}
\usepackage{listings}
\usepackage{bbm}
\usepackage{pgfplots}
\usepackage{color, colortbl}
\usepackage{array}
\usepackage{graphicx}
\usepackage{pifont}
\usepackage{multirow}
\newcommand{\xmark}{\ding{55}}
\usepackage{amsmath}      
\usepackage{amssymb}         
\usepackage{cancel}
\usepackage{dcolumn}
\usepackage{amsthm}
\usepackage[toc,page]{appendix}
\usepackage[font=small,labelfont=bf, skip=0pt]{caption}
\newcolumntype{H}{>{\setbox0=\hbox\bgroup}c<{\egroup}@{}}
\usepackage{makecell}

%you can link an external document called externalappendix.tex with the line below
%\externaldocument[SA-]{externalappendix}

\usepackage{amsthm} 
\usepackage[toc,page]{appendix}
\usepackage[font=small,labelfont=bf, skip=0pt]{caption}

\usepackage[a4paper,margin=1in,footskip=0.25in]{geometry}
\usepackage[authoryear,round]{natbib}
\bibliographystyle{plainnat}

\makeatletter
\def\thmhead@plain#1#2#3{%
	\thm@notefont{}% same as heading font
	\thmname{#1}\thmnumber{\@ifnotempty{#1}{ }\@upn{#2}}%
	\thmnote{ {\the\thm@notefont#3}}}
\let\thmhead\thmhead@plain
\itshape % body font
\makeatother

\newtheorem*{definition*}{Definition}
\newtheorem*{assumption*}{Assumption}
\newtheorem{assumption}{Assumption}
\newtheorem*{lemma*}{Lemma}
\newtheorem{lemma}{Lemma}
\newtheorem*{proposition*}{Proposition}
\newtheorem{proposition}{Proposition}
\newtheorem*{conjecture*}{Conjecture}
\newtheorem*{theorem*}{Theorem}
\newtheorem{theorem}{Theorem}
\newtheorem*{corollary*}{Corollary}
\newtheorem*{corollary}{Corollary}

\setcounter{MaxMatrixCols}{20}

\graphicspath{ {images/} }

\usepackage[many]{tcolorbox}
\newsavebox{\fmbox}
\newenvironment{fmpage}[1]
{\begin{lrbox}{\fmbox}\begin{minipage}{#1}}
	{\end{minipage}\end{lrbox}\fbox{\usebox{\fmbox}}}


\linespread{1.25}

\title{An Example Paper in \LaTeX}
\author{Leonard Goff\thanks{I thank person A and person B for their invaluable support on this project, and people C-Z for helpful discussions.}}
\date{Last updated: \today}

\begin{document}

\maketitle

\begin{abstract}
Here's my abstract. It's very small.
\end{abstract}

\large

%\tableofcontents

\section{Introduction} \label{secintro}
Since the dawn of civilization...

I will show in Section \ref{seclit} that blah blah blah

\section{Literature review} \label{seclit}

Some related papers here are \cite{Wilson2025} (which is the same paper as \citealt{Wilson2025}--see that I avoided nested parenthesis here). However, these papers leave open the important question of XYZ.

Here is a second paragraph.

\section{Data}
The primary data for this project are drawn from my anecdotal observations as a rider of the New York City subway.

\section{Empirical strategy}
My empirical strategy is based on the intuition that X, which I formalize with the following assumption
\begin{assumption}[(existence of the universe)] \label{idassumption}
	The universe exists.
\end{assumption}
\noindent We note that under Assumption \ref{idassumption}, we can make
\begin{align}
\gamma &= 3 \nonumber \\
	&= 2+1 \nonumber \\
	&= 1+1+1 \label{myeq}
\end{align}
\noindent This line is not indented. From Equation \ref{myeq} we can now motivate the final regression equation:
$$ y_{it} = \mu_i + \lambda_t + \rho d_i+ x_{it}'\beta + u_{it}$$
where under Assumption 1 we can interpret $\rho$ as the effect of $X$ on $Y$.\footnote{This is a GREAT footnote.}

\section{Results}
Table \ref{table1} reports the regression results.

\begin{table}[!htbp] \centering 
	
% Table created by stargazer v.5.2.2 by Marek Hlavac, Harvard University. E-mail: hlavac at fas.harvard.edu
% Date and time: Wed, Oct 30, 2019 - 11:15:58 AM
% Requires LaTeX packages: dcolumn 
\begin{tabular}{@{\extracolsep{5pt}}lD{.}{.}{-3} D{.}{.}{-3} D{.}{.}{-3} } 
\\[-1.8ex]\hline 
\hline \\[-1.8ex] 
 & \multicolumn{3}{c}{\textit{Dependent variable:}} \\ 
\cline{2-4} 
\\[-1.8ex] & \multicolumn{3}{c}{wage} \\ 
 & \multicolumn{1}{c}{not robust} & \multicolumn{1}{c}{robust} & \multicolumn{1}{c}{robust} \\ 
\\[-1.8ex] & \multicolumn{1}{c}{(1)} & \multicolumn{1}{c}{(2)} & \multicolumn{1}{c}{(3)}\\ 
\hline \\[-1.8ex] 
 employment & 0.023^{***} & 0.023^{***} & 0.023^{***} \\ 
  & (0.001) & (0.005) & (0.005) \\ 
  & & & \\ 
 unemp &  &  & -399.665^{***} \\ 
  &  &  & (50.053) \\ 
  & & & \\ 
 Constant & 23,452.380^{***} & 23,452.380^{***} & 25,647.870^{***} \\ 
  & (104.080) & (171.421) & (359.606) \\ 
  & & & \\ 
\hline \\[-1.8ex] 
Observations & \multicolumn{1}{c}{3,216} & \multicolumn{1}{c}{3,216} & \multicolumn{1}{c}{3,207} \\ 
R$^{2}$ & \multicolumn{1}{c}{0.206} & \multicolumn{1}{c}{0.206} & \multicolumn{1}{c}{0.227} \\ 
Adjusted R$^{2}$ & \multicolumn{1}{c}{0.205} & \multicolumn{1}{c}{0.205} & \multicolumn{1}{c}{0.226} \\ 
Residual Std. Error & \multicolumn{1}{c}{5,700.850 (df = 3214)} & \multicolumn{1}{c}{5,700.850 (df = 3214)} & \multicolumn{1}{c}{5,630.677 (df = 3204)} \\ 
F Statistic & \multicolumn{1}{c}{831.685$^{***}$ (df = 1; 3214)} & \multicolumn{1}{c}{831.685$^{***}$ (df = 1; 3214)} & \multicolumn{1}{c}{469.627$^{***}$ (df = 2; 3204)} \\ 
\hline 
\hline \\[-1.8ex] 
\textit{Note:}  & \multicolumn{3}{r}{$^{*}$p$<$0.1; $^{**}$p$<$0.05; $^{***}$p$<$0.01} \\ 
\end{tabular} 
	
	\caption{Here is a caption} \label{table1} 
\end{table}


\section{Conclusion} \label{secconclusion}
Here's my conclusion.

%References for things only mentioned in the supplemental appendix
%\nocite{someotherpaper}


\bibliography{biblio}

\begin{appendices}

	\section{First appendix} \label{appendix1}
	
	\section{Second appendix} \label{appendix2}

\end{appendices}

\end{document} 
